\documentclass[a4paper]{article}  

\newcommand{\Title}{Competitive Programming Note} 
\newcommand{\Author}{Moon Jam} 
\newcommand{\Date}{\today} 

\title{\Title} 
\author{\Author} 
\date{\Date}

\usepackage[margin=1in]{geometry}  
\usepackage{amsmath}
\usepackage{amssymb} 
\usepackage{fancyhdr}  
\usepackage{graphicx}
\usepackage{cancel} 
\usepackage{titlesec}
\usepackage{titling}
\usepackage[hidelinks]{hyperref}
\usepackage{CJKutf8} 
\usepackage{abstract}
\usepackage{indentfirst}
\usepackage{listings}

\lstset{language=C++}

\setlength{\parindent}{2em}

\pagestyle{fancy}
\fancyhead[LO,L]{\Author}
\fancyhead[CO,C]{\Title}
\fancyhead[RO,R]{\Date}
\fancyfoot[LO,L]{}
\fancyfoot[CO,C]{\thepage}
\fancyfoot[RO,R]{}
\renewcommand{\headrulewidth}{0.4pt}
\renewcommand{\footrulewidth}{0.4pt}

\pretitle{\begin{center}\LARGE\bfseries\vspace*{\fill}}
\posttitle{\end{center}\vskip 0.5em\vfill}

\begin{document}
\begin{CJK*}{UTF8}{bkai}

    \maketitle
    \thispagestyle{empty}
    \newpage

    \pagenumbering{roman}

    \newpage

    \tableofcontents

    \newpage

    \pagenumbering{arabic}

    \newpage
    \section{使用場景}
    \begin{itemize}
        \item 區間修改、區間查詢:線段樹+懶人標記
        \item 區間修改、單點查詢:BIT+差分、線段樹+懶人標記
        \item 單點修改、區間查詢:線段樹、BIT
        \item 區間修改、沒有查詢:差分
        \item 區間查詢、沒有修改:前綴和、稀疏表
    \end{itemize}

    \newpage
    \section{Segment Tree}
    \lstinputlisting{./code/segment_tree.cpp}

    \newpage
    \section{Binary Indexed Tree}
    \lstinputlisting{./code/BIT.cpp}

    \newpage
    \section{Sparse Table}

    \newpage
    \section{莫隊}
    \begin{itemize}
        \item 可以離線處理(沒有修改)
        \item 區間眾數、區間最大連續和、區間某個數字的數量
    \end{itemize}
    \href{https://atcoder.jp/contests/abc242/tasks/abc242_g}{Range Pairing Query}
    \lstinputlisting{./code/Range Pairing Query.cpp}

    \newpage
    \section{DSU}

    \newpage
    \section{PBDS \& ROPE}
    \lstinputlisting{./code/pbds.cpp}

    \hrule

    \lstinputlisting{./code/rope.cpp}

    \newpage
    \section{拓樸排序}


    \newpage
    \section{Floyd}


    \newpage
    \section{Dijkstra}


    \newpage
    \section{Bellman-Ford}


    \newpage
    \section{Prime}


    \newpage
    \section{Kruskal}


    \newpage
    \section{Tarjan}

    \newpage
    \section{Kosaraju}
    \lstinputlisting{./code/kosaraju.cpp}
    \subsection{2-SAT}
    Each pair ($a_i \lor b_i$) generates two edges: $\lnot a_i \rightarrow b_i$ and $\lnot b_i \rightarrow a_i$
    \begin{enumerate}
        \item find SCC of G (use kosaraju can find topological ordering at the same time)
        \item for all $x$ : \\
          if($x$ and $\lnot x$ in the same SCC) return IMPOSSIBLE
        \item for all $x$ :\\
          if(topological ordering $\lnot x > x$) set $x$ to true\\
          else set $x$ to false
    \end{enumerate}

    \newpage

    \section{Eulerian Path}
    \subsection{雙向}
    \lstinputlisting{./code/Eulerian_path.cpp}
    \subsection{單向}
    \lstinputlisting{./code/Eulerian_path_one.cpp}
    \subsection{De Bruijn Sequence}
    \lstinputlisting{./code/De Bruijn Sequence.cpp}

    \newpage
    \section{凸包}


\end{CJK*}
\end{document}